%%%%%%%%%%%%%%%%%%%%%%%%%%%%%%%%%%%%%%%%%%%%%%%%%%%%%%%%%%%%%%%%%%
%%%%%%%%%%%%%%%%%%%%%%%%%%%%%%%%%%%%%%%%%%%%%%%%%%%%%%%%%%%%%%%%%%
%Packages
\documentclass{yourmommaaintgotno}
\setHW{5}

\begin{document}

\begin{problem}{TYPE another PROBLEM HERE.}
And your answer here
\end{problem}

\begin{mproblem}{TYPE YOUR main PROBLEM HERE.}
  \item and your sub-questions
  \item here
  \item and
  \item here
\end{mproblem}

\begin{Exercise}
  TYPE YOUR PROBLEM HERE.
\end{Exercise}
\begin{Solution}
Personally I recommend Mathpix (https://mathpix.com/),
which can easily export your ProblemBook.pdf to \LaTeX \ code.
\end{Solution}

\begin{exercise}{TITLE}{something}
  TYPE YOUR PROBLEM HERE.
\end{exercise}
\begin{solution}
Personally I recommend Mathpix (https://mathpix.com/),
which can easily export your ProblemBook.pdf to \LaTeX \ code.
\end{solution}

\begin{exercise}
  TYPE YOUR PROBLEM HERE.
\end{exercise}
\begin{Solution}
Personally I recommend Mathpix (https://mathpix.com/),
which can easily export your ProblemBook.pdf to \LaTeX \ code.
\end{Solution}


\begin{problem}{Here is a problem}
  and another
\end{problem}
\end{document}
